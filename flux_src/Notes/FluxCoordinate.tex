\documentclass[12pt,prb,aps,notitlepage]{revtex4-1}
\usepackage{epsf}
\usepackage{rotate}
\usepackage{amssymb}
\usepackage{amsmath}
\newcommand {\bxi} {\mbox{\boldmath$\xi$}}
\newcommand {\bPsi} {\mbox{\boldmath$\Psi$}}
\newcommand {\bDelta} {\mbox{\boldmath$\Delta$}}
\newcommand {\bGamma} {\mbox{\boldmath$\Gamma$}}

\begin{document}
\title{Construction of Flux Coordinate System}
\maketitle

\section{Construction of Coordinate System}
Let $R$, $\phi$, $Z$ be conventional right-handed cylindrical coordinates. Let us adopt
a normalization scheme in which all lengths are normalized to $R_0$, all magnetic field-strengths to $B_0$, and all pressures to $B_0^{\,2}/\mu_0$. In the following, all quantities are assumed to be normalized. 
We can write the equilibrium magnetic field in the form
\begin{equation}
{\bf B} = \nabla\phi\times \nabla\psi + g(\psi)\,\nabla\phi = \nabla[\phi-q(\psi)\,\theta]\times\nabla\psi,
\end{equation}
where the poloidal flux, $\psi(R,Z)$,  and the toroidal flux function, $g(\psi)$, are both given, and 
\begin{equation}
\nabla\psi\times\nabla\theta\cdot\nabla \phi = \frac{g}{q\,R^{\,2}}.
\end{equation}
The equilibrium is assumed to be up-down symmetric, so that $\psi(R,-Z)=\psi(R,Z)$ for all $R$ and $Z$. 
Here, $\theta$ is a so-called  ``straight" poloidal angle, and $q(\psi)$ is the safety-factor. 
Suppose that the magnetic axis (where $\psi_R=\psi_Z=0$) lies at $R=R_c$, $Z=0$. 
Let ${\mit\Psi}(R,Z)=\psi(R,Z)/\psi_c$, where $\psi_c=\psi(R_c,0)$. 
The previous equation implies that
\begin{equation}
\frac{d\theta}{dl} = \frac{g}{q}\,\frac{1}{|\psi_c|\,R\,\sqrt{{\mit\Psi}_R^{\,2}+{\mit\Psi}_Z^{\,2}}},
\end{equation}
where $l$ represents distance along a constant-${\mit\Psi}$ surface. The corresponding increments of $R$ and $Z$
are
\begin{align}
dR &= -\frac{{\mit\Psi}_Z\,dl}{\sqrt{{\mit\Psi}_R^{\,2} + {\mit\Psi}_Z^{\,2}}},\\[0.5ex]
dZ &= \frac{{\mit\Psi}_R\,dl}{\sqrt{{\mit\Psi}_R^{\,2} + {\mit\Psi}_Z^{\,2}}},
\end{align}
respectively.
Since $\theta$ must increase by $2\pi$ in a circuit around a given flux surface, we can write
\begin{equation}
\frac{q({\mit\Psi})}{g({\mit\Psi})} = \frac{1}{2\pi\,|\psi_c|}\oint \frac{dl}{R\,\sqrt{{\mit\Psi}_R^{\,2}+{\mit\Psi}_Z^{\,2}}},
\end{equation}
which determines $q({\mit\Psi})$. We can then calculate $\theta$ from 
\begin{align}
\frac{dR}{d\theta} &= -|\psi_c|\,\frac{q({\mit\Psi})}{g({\mit\Psi})}\,R\,{\mit\Psi}_Z,\\[0.5ex]
\frac{dZ}{d\theta} &= |\psi_c|\,\frac{q({\mit\Psi})}{g({\mit\Psi})}\,R\,{\mit\Psi}_R.
\end{align}

It is convenient to define a new flux-surface label, $r$, which is such that
$r=0$ on the magnetic axis, $r=1$ at the plasma boundary, and
\begin{equation}
\nabla r\times \nabla\theta\cdot\nabla\phi = \frac{1}{\epsilon_\ast^{\,2}\,r\,R^{\,2}}.
\end{equation}
It follows that
\begin{equation}
\epsilon_\ast =\left( 2\,|\psi_c|\int_0^1\frac{q({\mit\Psi})}{g({\mit\Psi})}\,d{\mit\Psi}\right)^{1/2},
\end{equation}
and
\begin{equation}
r({\mit\Psi})= \left(\frac{2\,|\psi_c|}{\epsilon_\ast^{\,2}}\int_{\mit\Psi}^1 \frac{q({\mit\Psi}')}{g({\mit\Psi}')}\,d{\mit\Psi}'\right)^{1/2}.
\end{equation}

Let $R_r=\left.\partial R/\partial r\right|_\theta$, etc. It follows that
\begin{align}
\epsilon_\ast^{\,2}\,r\,R&= R_\theta\,Z_r-R_r\,Z_\theta,\\[0.5ex]
|\nabla r|^{-2} &= \frac{\epsilon_\ast^{\,4}\,r^2\,R^{\,2}}{R_\theta^{\,2} + Z_\theta^{\,2}},\\[0.5ex]
\frac{r\,\nabla r\cdot\nabla\theta}{|\nabla r|^{\,2}} &= -\frac{r\,(R_r\,R_\theta + Z_r\,Z_\theta)}{R_\theta^{\,2}+Z_\theta^{\,2}}.
\end{align}

\section{Data Required by {\sc TOMUHAWC}}
All lengths in {\sc TOMUHAWC} are normalized to $\epsilon_\ast\,R_0$. 
Let
\begin{align}
R^{\,2}&= \sum_{k=0,\infty} M^{(1)}_k(r)\,\cos(k\,\theta),\\[0.5ex]
\epsilon_\ast^{\,-2}\,|\nabla r|^{\,-2}\,R^{\,-2}&= \sum_{k=0,\infty} M^{(2)}_k(r)\,\cos(k\,\theta),\\[0.5ex]
\epsilon_\ast^{\,-2}\,|\nabla r|^{\,-2}&= \sum_{k=0,\infty} M^{(3)}_k(r)\,\cos(k\,\theta),\\[0.5ex]
\epsilon_\ast^{\,-2}\,|\nabla r|^{\,-2}\,R^{\,2} &= \sum_{k=0,\infty} M^{(4)}_k(r)\,\cos(k\,\theta),\\[0.5ex]
\epsilon_\ast^{\,-2}\,|\nabla r|^{\,-2}\,R^{\,4}  &= \sum_{k=0,\infty} M^{(5)}_k(r)\,\cos(k\,\theta),\\[0.5ex]
(r\,\nabla r\cdot\nabla \theta)\, |\nabla r|^{\,-2}&= \sum_{k=1,\infty} M^{(6)}_k(r)\,\sin(k\,\theta),\\[0.5ex]
(r\,\nabla r\cdot\nabla \theta)\, |\nabla r|^{\,-2}\,R^{\,2}&= \sum_{k=1,\infty} M^{(7)}_k(r)\,\sin(k\,\theta),\\[0.5ex]
\epsilon_\ast^{\,2}\,|\nabla r|^{\,2}&= \sum_{k=0,\infty} M^{(8)}_k(r)\,\cos(k\,\theta),\\[0.5ex]
R^{\,4}&= \sum_{k=0,\infty} M^{(9)}_k(r)\,\cos(k\,\theta).
\end{align}
The {\sc TOMUHAWC} code requires the $M^{(1)}_k(r)$, $M^{(2)}_k(r)$, $M^{(3)}_k(r)$, $M^{(4)}_k(r)$,
$M^{(5)}_k(r)$, $M^{(6)}_k(r)$, and $M^{(7)}_k(r)$, as well as $M^{(8)}_0(r)$ and $M^{(9)}_0(r)$. The code also needs 
$q(r)$, $dq(r)/dr$, and the following six profile
functions:
\begin{align}
p_1(r) &= \epsilon_\ast^{\,2}\,r^{\,2},\\[0.5ex]
p_2(r) &=\frac{dg}{d\psi},\\[0.5ex]
p_3(r) &= \frac{q}{g}\,\frac{dP}{d\psi},\\[0.5ex]
p_4(r) &= -\frac{d\ln(q/g)}{d\ln r},\\[0.5ex]
p_5(r) & = \left(\epsilon_\ast\,r\,\frac{g}{q}\right)^2,\\[0.5ex]
p_6(r) &= {\mit\Gamma}\,P,
\end{align}
where $P(r)$ is the pressure profile, and ${\mit\Gamma}$ the plasma ratio of specific heats. 
All functions are required in the range $0\leq r\leq 1$. 

\end{document}